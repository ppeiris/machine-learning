\chapter{Problem Statement}

Kepler is a single instrument spacecraft that collect most contiguous and long-running photometric time series possible. Kepler has the capability to observe approximately 170,000 stars simultaneously while it is operating. The fundamental objective of the Kepler mission is to detect a large number of transiting exoplanets. The ultimate goal of the primary mission was a characterize the frequency of exoplanets on diameter, orbital period and host star.  The Manual classification of the findings of Kepler object has proven very time-consuming. The new space-based, transit photometry missions such as K2 \cite{2014PASP..126..398H}, TESS \cite{2014SPIE.9143E..20R}, and PLATO 2.0 \cite{2014ExA....38..249R} also produce a large number of the dataset that demands some level of automation to do the classification. 

Using machine learning classification techniques, we can speed up the process and provide a more continuous rating of planarity candidates. There are various of machine learning classification techniques has been applied to Kepler dataset including random forests, SVM, K-mean clustering \cite{2015ApJ...800...99T, 2015ApJ...806....6M}.  In this project, we are attempting to train a Multilayered Neural Network to identify the potential planetary candidates in the Kepler dataset. 
